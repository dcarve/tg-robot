

\chapter{Próximas etapas}

\section{Montagem do robô}
Nesta etapa, será realizada a montagem mecânica do robô, com as ligações
elétricas e mecânicas dos componentes segundo as especificações apresentadas
até o momento.

\section{Calibração de controle nos atuadores}
Nesta etapa, busca-se verificar se os parâmetros especificados para o controle
do robô estão corretos - mais especificamente, busca-se saber se é possível
controlar o robô diretamente, modificando sua velocidade, forçando-o a fazer
curvas, testando sua holonomicidade, e afins.

\section{Mapeamento de ambientes}
Nesta etapa, busca-se imbuir o robô da capacidade de receber mapas de ambientes
internos em que ele possa operar. Procura-se aqui verificar se o formato de
entrada é adequado e compatível com as habilidades de movimentação do robô.

\section{Calibração de controle automático}
Nesta etapa, o objetivo é definir corretamente os parâmetros de controle
automático a operação do robô, correção de rotas, e similares. Espera-se fazer
uso de um controlador PID para isso.

\section{Testes autônomos}
Nesta etapa, busca-se integrar as atividades realizadas até este momento,
aferindo a capacidade do robô de se ater a rotas pré-determinadas, de acordo com
o mapeamento do ambiente que lhe foi provido, a fim de atingir aos objetivos 
propostos deste trabalho.