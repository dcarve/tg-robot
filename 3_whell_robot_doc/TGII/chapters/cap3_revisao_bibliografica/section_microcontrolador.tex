Microcontroladores são circuitos integrados compactos desenvolvidos para governar
uma operação específica em um sistema embarcado. No contexto da aplicação deste
trabalho, o uso de um microcontrolador é fundamental para se obter o controle
desejado de trajetória e posicionamento do robô.

STM32 é uma família de microcontroladores de 32-bits fabricados pela
STMicroelectronics. O processador empregado nessa família é o ARM Cortex,
baseado em arquitetura Harvard, de 32-bits. Os microcontroladores STM32 fornecem
base para uma uma grande variedades de sistemas embarcados, com custo inferior
e maior flexibilidade quando comparado ao Arduino com ATmega, que possui
microcontroladores de 8 a 16 bits. Contudo, essa flexibilidade e baixo custo têm
como contrapartida o requerimento de um maior nível de experiência em
programação C do que o necessário para desenvolver as mesmas soluções em
Arduino (cuja concepção teve como objetivo maior acessibilidade para iniciantes
em programação em geral, e também  em desenvolvimento de aplicações com
microcontroladores \cite{cortex_m3}).

