CAD é a sigla para "computer-aided design", design auxiliado por computador, e pode ser definido como o uso de sistemas computadorizados (hardware + software)
para criação, modificação, análise ou otimização do design \cite{computer_aided_esign}.
O software CAD consiste em um programa capaz de implementar gráficos computadorizados e aplicar funções como análise tensão-deformação de componentes, resposta dinâmica de mecanismos,
cálculos de transferência de calor. As aplicações variam conforme a necessidade da área.

\subsection{AutoCAD}
AutoCAD foi criado em 1982, seu uso mais popular é em desenhos arquitetônicos, mas também tem um forte uso na criação de desenhos técnicos.

\begin{figure}[h]
	\centering
	\includegraphics[width=1\textwidth]{figures/autocad_screen}
	\caption{Interface do AutoCad}
	\label{fig:interface_autocad}
\end{figure}

\subsection{SolidWorks CAD 3D}
Com um foco maior em projetos de engenharia, esse software CAD tem um foco na criação do modelo 3D das peças, considerando precisão de medidas e materiais.
O pacote de softwares adicionais oferece simulações de elementos finitos como análise térmica, vibrações, queda, dinâmica, pressão entre outros.

\begin{figure}[h]
	\centering
	\includegraphics[width=1\textwidth]{figures/soliworks}
	\caption{Interface do SolidWorks}
	\label{fig:interface_soliwoks}
\end{figure}