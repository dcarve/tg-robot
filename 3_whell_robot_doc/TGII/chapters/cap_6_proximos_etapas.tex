

\chapter{Próximas etapas}

\section{Controle PID em motor DC}
Após lidar com o problema da não linearidade da respostado do motor ao sinal PWM
o próximo passo em controlar a velocidade é a implementação do controle PID.
O desafio será controlar a posição no tempo, considerando a velocidade resultante do robô.

\section{Integração bluetooth}
Para testar a cinematica do robô, iremos controla-lo via bluetooth. Para isso falta implementar a leitura via serial usando protocolo UART.

\section{Teste com motor de passo de alto torque - bj42d15}
Foi considerado trocar o tipo de motor e fazer um compartivo entre os diferentes tipos de controle e implementação necessário para cada tipo de motor,
e apontar as diferenças de resultados.
Foi sugerido usar o motor de passo bj42d15, comum em impressoras 3d, que possui um torque mais alto e bons resultados no uso de equipamentos que exigem precisão de posicionamento.