
% ---
% Capa
% ---
\imprimircapa
% ---

% ---
% Folha de rosto
% (o * indica que haverá a ficha bibliográfica)
% ---
\imprimirfolhaderosto
% ---


% ---
% RESUMOS
% ---

% resumo em português
\setlength{\absparsep}{18pt} % ajusta o espaçamento dos parágrafos do resumo
\begin{resumo}
 Segundo a \citeonline[3.1-3.2]{NBR6028:2003}, o resumo deve ressaltar o
 objetivo, o método, os resultados e as conclusões do documento. A ordem e a extensão
 destes itens dependem do tipo de resumo (informativo ou indicativo) e do
 tratamento que cada item recebe no documento original. O resumo deve ser
 precedido da referência do documento, com exceção do resumo inserido no
 próprio documento. (\ldots) As palavras-chave devem figurar logo abaixo do
 resumo, antecedidas da expressão Palavras-chave:, separadas entre si por
 ponto e finalizadas também por ponto.

 \textbf{Palavras-chave}: latex. abntex. editoração de texto.
\end{resumo}

% resumo em inglês
\begin{resumo}[Abstract]
 \begin{otherlanguage*}{english}
   This is the english abstract.

   \vspace{\onelineskip}
 
   \noindent 
   \textbf{Keywords}: latex. abntex. text editoration.
 \end{otherlanguage*}
\end{resumo}


% ---
% inserir lista de ilustrações
% ---
\pdfbookmark[0]{\listfigurename}{lof}
\listoffigures*
\cleardoublepage
% ---

% ---
% inserir lista de quadros
% ---
\pdfbookmark[0]{\listofquadrosname}{loq}
\listofquadros*
\cleardoublepage
% ---

% ---
% inserir lista de tabelas
% ---
\pdfbookmark[0]{\listtablename}{lot}
\listoftables*
\cleardoublepage
% ---

% ---
% inserir lista de abreviaturas e siglas
% ---
\begin{siglas}
  \item[ABNT] Associação Brasileira de Normas Técnicas
  \item[abnTeX] ABsurdas Normas para TeX
\end{siglas}
% ---

% ---
% inserir lista de símbolos
% ---
\begin{simbolos}
  \item[$ \omega $] Letra grega minúscula ômega
  \item[$ \theta $] Letra grega minúscula theta
\end{simbolos}
% ---

% ---
% inserir o sumario
% ---
\pdfbookmark[0]{\contentsname}{toc}
\tableofcontents*
\cleardoublepage
% ---
