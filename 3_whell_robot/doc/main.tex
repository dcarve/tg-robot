
\documentclass[
	% -- opções da classe memoir --
	12pt,				% tamanho da fonte
	openright,			% capítulos começam em pág ímpar (insere página vazia caso preciso)
	oneside,			% para impressão em recto e verso. Oposto a oneside
	a4paper,			% tamanho do papel. 
	% -- opções da classe abntex2 --
	%chapter=TITLE,		% títulos de capítulos convertidos em letras maiúsculas
	%section=TITLE,		% títulos de seções convertidos em letras maiúsculas
	%subsection=TITLE,	% títulos de subseções convertidos em letras maiúsculas
	%subsubsection=TITLE,% títulos de subsubseções convertidos em letras maiúsculas
	% -- opções do pacote babel --
	english,			% idioma adicional para hifenização
	french,				% idioma adicional para hifenização
	spanish,			% idioma adicional para hifenização
	brazil				% o último idioma é o principal do documento
	]{abntex2}




% Pacotes básicos 
\usepackage{lmodern}			% Usa a fonte Latin Modern			
\usepackage[T1]{fontenc}		% Selecao de codigos de fonte.
\usepackage[utf8]{inputenc}		% Codificacao do documento (conversão automática dos acentos)
\usepackage{indentfirst}		% Indenta o primeiro parágrafo de cada seção.
\usepackage{color}				% Controle das cores
\usepackage{graphicx}			% Inclusão de gráficos
\usepackage{microtype} 			% para melhorias de justificação
\usepackage{amsmath}			% lib para equacoes
\usepackage{svg}
%\usepackage{float}
\usepackage{graphicx}
% ---
		
% ---
% Pacotes adicionais, usados apenas no âmbito do Modelo Canônico do abnteX2
% ---
\usepackage{lipsum}				% para geração de dummy text
% ---

% ---
% Pacotes de citações
% ---
\usepackage[brazilian,hyperpageref]{backref}	 % Paginas com as citações na bibl
\usepackage[alf]{abntex2cite}	% Citações padrão ABNT

% --- 
% CONFIGURAÇÕES DE PACOTES
% --- 

% ---
% Configurações do pacote backref
% Usado sem a opção hyperpageref de backref
\renewcommand{\backrefpagesname}{Citado na(s) página(s):~}
% Texto padrão antes do número das páginas
\renewcommand{\backref}{}
% Define os textos da citação
\renewcommand*{\backrefalt}[4]{
	\ifcase #1 %
		Nenhuma citação no texto.%
	\or
		Citado na página #2.%
	\else
		Citado #1 vezes nas páginas #2.%
	\fi}%
% ---

% ---
% Informações de dados para CAPA e FOLHA DE ROSTO
% ---
\titulo{Robô omnidirecional de 3 rodas}
\autor{Daniel Ermelino Carvalho \\ Lucas Lima Pereira}
\local{Brasil}
\data{2022}
\orientador{Lauro César Araujo}
\instituicao{%
  Universidade Federal do ABC
  \par
  CECS
  \par
   Engenharia de Instrumentação, Automação e Robótica}
\tipotrabalho{Tese (Doutorado)}
% O preambulo deve conter o tipo do trabalho, o objetivo, 
% o nome da instituição e a área de concentração 
\preambulo{Modelo canônico de trabalho monográfico acadêmico em conformidade com
as normas ABNT apresentado à comunidade de usuários \LaTeX.}
% ---


% ---
% Configurações de aparência do PDF final

% alterando o aspecto da cor azul
\definecolor{blue}{RGB}{5,5,180}

% informações do PDF
\makeatletter
\hypersetup{
     	%pagebackref=true,
		pdftitle={\@title}, 
		pdfauthor={\@author},
    	pdfsubject={\imprimirpreambulo},
	    pdfcreator={LaTeX with abnTeX2},
		pdfkeywords={abnt}{latex}{abntex}{abntex2}{trabalho acadêmico}, 
		colorlinks=true,       		% false: boxed links; true: colored links
    	linkcolor=blue,          	% color of internal links
    	citecolor=blue,        		% color of links to bibliography
    	filecolor=magenta,      		% color of file links
		urlcolor=blue,
		bookmarksdepth=4
}
\makeatother
% --- 

% ---
% Posiciona figuras e tabelas no topo da página quando adicionadas sozinhas
% em um página em branco. Ver https://github.com/abntex/abntex2/issues/170
\makeatletter
\setlength{\@fptop}{5pt} % Set distance from top of page to first float
\makeatother
% ---

% ---
% Possibilita criação de Quadros e Lista de quadros.
% Ver https://github.com/abntex/abntex2/issues/176
%
\newcommand{\quadroname}{Quadro}
\newcommand{\listofquadrosname}{Lista de quadros}

\newfloat[chapter]{quadro}{loq}{\quadroname}
\newlistof{listofquadros}{loq}{\listofquadrosname}
\newlistentry{quadro}{loq}{0}

% configurações para atender às regras da ABNT
\setfloatadjustment{quadro}{\centering}
\counterwithout{quadro}{chapter}
\renewcommand{\cftquadroname}{\quadroname\space} 
\renewcommand*{\cftquadroaftersnum}{\hfill--\hfill}

\setfloatlocations{quadro}{hbtp} % Ver https://github.com/abntex/abntex2/issues/176
% ---

% --- 
% Espaçamentos entre linhas e parágrafos 
% --- 

% O tamanho do parágrafo é dado por:
\setlength{\parindent}{1.3cm}

% Controle do espaçamento entre um parágrafo e outro:
\setlength{\parskip}{0.2cm}  % tente também \onelineskip

% ---
% compila o indice
% ---
\makeindex
% ---

% ----
% Início do documento
% ----
\begin{document}

% Seleciona o idioma do documento (conforme pacotes do babel)
%\selectlanguage{english}
\selectlanguage{brazil}

% Retira espaço extra obsoleto entre as frases.
\frenchspacing 

% ----------------------------------------------------------
% ELEMENTOS PRÉ-TEXTUAIS
% ----------------------------------------------------------
% \pretextual

% ---
% Capa
% ---
\imprimircapa
% ---

% ---
% Folha de rosto
% (o * indica que haverá a ficha bibliográfica)
% ---
\imprimirfolhaderosto
% ---


% ---
% RESUMOS
% ---

% resumo em português
\setlength{\absparsep}{18pt} % ajusta o espaçamento dos parágrafos do resumo
\begin{resumo}
 Segundo a \citeonline[3.1-3.2]{NBR6028:2003}, o resumo deve ressaltar o
 objetivo, o método, os resultados e as conclusões do documento. A ordem e a extensão
 destes itens dependem do tipo de resumo (informativo ou indicativo) e do
 tratamento que cada item recebe no documento original. O resumo deve ser
 precedido da referência do documento, com exceção do resumo inserido no
 próprio documento. (\ldots) As palavras-chave devem figurar logo abaixo do
 resumo, antecedidas da expressão Palavras-chave:, separadas entre si por
 ponto e finalizadas também por ponto.

 \textbf{Palavras-chave}: latex. abntex. editoração de texto.
\end{resumo}

% resumo em inglês
\begin{resumo}[Abstract]
 \begin{otherlanguage*}{english}
   This is the english abstract.

   \vspace{\onelineskip}
 
   \noindent 
   \textbf{Keywords}: latex. abntex. text editoration.
 \end{otherlanguage*}
\end{resumo}


% ---
% inserir lista de ilustrações
% ---
\pdfbookmark[0]{\listfigurename}{lof}
\listoffigures*
\cleardoublepage
% ---

% ---
% inserir lista de quadros
% ---
\pdfbookmark[0]{\listofquadrosname}{loq}
\listofquadros*
\cleardoublepage
% ---

% ---
% inserir lista de tabelas
% ---
\pdfbookmark[0]{\listtablename}{lot}
\listoftables*
\cleardoublepage
% ---

% ---
% inserir lista de abreviaturas e siglas
% ---
\begin{siglas}
  \item[ABNT] Associação Brasileira de Normas Técnicas
  \item[abnTeX] ABsurdas Normas para TeX
\end{siglas}
% ---

% ---
% inserir lista de símbolos
% ---
\begin{simbolos}
  \item[$ \omega $] Letra grega minúscula ômega
  \item[$ \theta $] Letra grega minúscula theta
\end{simbolos}
% ---

% ---
% inserir o sumario
% ---
\pdfbookmark[0]{\contentsname}{toc}
\tableofcontents*
\cleardoublepage
% ---



% ----------------------------------------------------------
% ELEMENTOS TEXTUAIS
% ----------------------------------------------------------
\textual

% ----------------------------------------------------------
% Introdução (exemplo de capítulo sem numeração, mas presente no Sumário)
% ----------------------------------------------------------
\chapter{Introdução}


Hello, here is some text without a meaning...



% Capitulo com exemplos de comandos inseridos de arquivo externo 
\include{abntex2-modelo-include-comandos}

\chapter{Robô de 3 rodas}\label{cap_trabalho_academico}

\section{Robô}
robo etc.

\section{Modelo matemático}

\begin{gather}
	\begin{bmatrix} \dot{V}{w1} \\  \dot{V}{w2} \\  \dot{V}{w3} \end{bmatrix}
	=
	\begin{bmatrix}
		0 & 2/3 & 1/3 \\
		-1/\sqrt{3} & -1/3 & 1/3\\
		1/\sqrt{3} & -1/3 & 1/3
	\end{bmatrix}
	\cdot
	\begin{bmatrix} \dot{V}\cdot \cos{\theta} \\  \dot{V}\cdot \sin{\theta} \\  \dot{\omega} \end{bmatrix}
   \end{gather}

\chapter{Outro Capítulo}

\section{Aliquam vestibulum fringilla lorem}


\lipsum[1]

\lipsum[2-3]




% ----------------------------------------------------------
% Finaliza a parte no bookmark do PDF
% para que se inicie o bookmark na raiz
% e adiciona espaço de parte no Sumário
% ----------------------------------------------------------
\phantompart

% ---
% Conclusão
% ---
\chapter{Conclusão}
% ---

\lipsum[31-33]

% ----------------------------------------------------------
% ELEMENTOS PÓS-TEXTUAIS
% ----------------------------------------------------------
\postextual
% ----------------------------------------------------------

% ----------------------------------------------------------
% Referências bibliográficas
% ----------------------------------------------------------
\bibliography{abntex2-modelo-references}


% Glossário
\input{glossary/glossary_class.tex}


% Apêndices
\begin{apendicesenv}

% Imprime uma página indicando o início dos apêndices
\partapendices


\chapter{apendice 1}


\lipsum[1]

\chapter{apendice 2}

\lipsum[1]


\end{apendicesenv}

% Anexos
\begin{anexosenv}

% Imprime uma página indicando o início dos anexos
\partanexos


\begin{anexosenv}

\partanexos

\chapter{Cálculo modelo de 3 rodas}

\begin{figure}[h]
	\centering
	\includegraphics{figures/digram_model_dedution}
	\caption{diagrama do modelo - dedução da matriz}
	\label{lof}
\end{figure}

\begin{equation}
    \begin{split}
        \overrightarrow{V}_{l} = 
        \overrightarrow{V}_{w1}
        + \overrightarrow{V}_{w2}
        + \overrightarrow{V}_{w3}
    \end{split}
\end{equation}

\begin{equation}
    \begin{split}
        \overrightarrow{\omega} = 
        \frac{\vert\overrightarrow{V}_{w1}\vert}{L}
        + \frac{\vert\overrightarrow{V}_{w2}\vert}{L}
        + \frac{\vert\overrightarrow{V}_{w3}\vert}{L}
    \end{split}
\end{equation}


\begin{gather*}
        V_{l} \angle \theta =  
        V_{w1} \angle \left(-\frac{\pi}{2}\right) 
        + V_{w2} \angle \left(\frac{2\pi}{3}-\frac{\pi}{2}\right) 
        + V_{w3} \angle \left(\frac{4\pi}{3}-\frac{\pi}{2}\right) 
\end{gather*}

\begin{align*}
    V_{l} \cos{ \theta } + jV_{l} \sin{\theta} =  
    V_{w1} \cos{ \left(-\frac{\pi}{2}\right)} + jV_{w1} \sin{ \left(-\frac{\pi}{2}\right) } \\
    + V_{w2}  \cos{ \left(\frac{\pi}{6}\right) } + jV_{w2}  \sin{ \left(\frac{\pi}{6}\right) }  \\
    + V_{w3} \cos{ \left(\frac{5\pi}{6}\right) } + jV_{w2}  \sin{ \left(\frac{5\pi}{6}\right) } 
\end{align*}

\begin{equation*}
    \begin{split}
        \omega = 
        \frac{V_{w1}}{L}
        + \frac{V_{w2}}{L}
        + \frac{V_{w3}}{L}
    \end{split}
\end{equation*}


\begin{gather}
	\begin{bmatrix} V\cdot \cos{\theta} \\  V\cdot \sin{\theta} \\  \omega \end{bmatrix}
	=
	\begin{bmatrix}
		\cos{\left(-\frac{\pi}{2}\right)} & \cos{\left(\frac{\pi}{6}\right)} & \cos{\left(\frac{5\pi}{6}\right)} \\
		\sin{\left(-\frac{\pi}{2}\right)} & \sin{\left(\frac{\pi}{6}\right)} & \sin{\left(\frac{5\pi}{6}\right)} \\
		\frac{1}{L} & \frac{1}{L} & \frac{1}{L}
	\end{bmatrix}
	\cdot
	\begin{bmatrix} V_{w1} \\  V_{w2} \\  V_{w3} \end{bmatrix}
\end{gather}



\begin{gather}
	\begin{bmatrix} V_{w1} \\  V_{w2} \\  V_{w3} \end{bmatrix}
	=
	\begin{bmatrix}
		0 & 2/3 & L/3 \\
		-1/\sqrt{3} & -1/3 & L/3\\
		1/\sqrt{3} & -1/3 & L/3
	\end{bmatrix}
	\cdot
	\begin{bmatrix} V\cdot \cos{\theta} \\  V\cdot \sin{\theta} \\  \omega \end{bmatrix}
\end{gather}

\end{anexosenv}

\chapter{Outro Anexo}


\lipsum[32]



\end{anexosenv}

%---------------------------------------------------------------------
% INDICE REMISSIVO
%---------------------------------------------------------------------
\phantompart
\printindex
%---------------------------------------------------------------------

\end{document}