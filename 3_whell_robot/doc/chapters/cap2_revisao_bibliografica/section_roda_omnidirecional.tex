A roda omnidirecional aparece em vários modelos na literatura, como exemplo o design feito por J. Graboweicki em 1919 \cite{patent_US1305535A}
o design feito por Josef Blumrich em 1972 \cite{patent_US3789947A}.
A roda consiste em rolos perpendiculares (90°) a direção de giro da roda,o efeito é sua capacidade da roda se mover em mais de uma direção ao mesmo tempo.

\begin{figure}[h]
	\centering
	\includegraphics{figures/omniwheel}
	\caption{Modelo de uma Omniwheel \cite{draw_omniwheel}}
\end{figure}

A relação entre velocidade linear e angular da roda, se da por:

\[V_{w1} = \omega_{w1}\cdot r \] 

$V_{w}$ é velocidade linear da roda, $r$ raio da roda, $\omega_{w} $ é a velocidade angular da roda.


Uma variação da roda omnidirecional é a roda mecanum, inventada por Bengt Ilon \cite{patent_US3876255A}, que possue rolos em 45°.

