
\chapter{Revisão Bibliografica}

\section{Veculo omnidirecional de 3 rodas}

\section{Roda Omnidirecional}

A roda omnidirecional aparece em vários modelos na literatura, como exemplo o design feito por J. Graboweicki em 1919 \cite{patent_US1305535A}
o design feito por Josef Blumrich em 1972 \cite{patent_US3789947A}
O funcionamento basico da roda é sua capacidade de se mover em mais de uma direção ao mesmo tempo, 


\section{Roda Omnidirecional}




%Bengt Erland Ilon. Wheels for a course stable selfpropelling vehicle movable
%any desired direction on the ground or some other base, 1975. US Patent No.
%3876255.

\section{STM32F103C8}


\section{Motor com encoder}


\section{Driver Motor Ponte H L298n}


\section{Comunicação serial}


\section{controle PDI}


\section{filtro digital}




