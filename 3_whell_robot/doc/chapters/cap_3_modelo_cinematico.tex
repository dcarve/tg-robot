
\chapter{Modelagem Cinemática}

\section{Dedução da matriz por cinemática direta}

$\overrightarrow{V}$ é o vetor de velocidade linear do robô, $V_{w1}$, $V_{w2}$, $V_{w3}$ são as velocidades lineares das rodas 1,2,3. 
$\omega $ é a velocidade angular do robô a partir do centro geometrico do robô.
$L$ é a distância entre o centro de geometrico da roda e o centro de geometrico do robô.



\begin{figure}[h]
	\centering
	\includegraphics{figures/digram_model_dedution}
	\caption{Diagrama do robô}
\end{figure}

\begin{equation}
    \begin{split}
        \overrightarrow{V}_{l} = 
        \overrightarrow{V}_{w1}
        + \overrightarrow{V}_{w2}
        + \overrightarrow{V}_{w3}
    \end{split}
\end{equation}

\begin{equation}
    \begin{split}
        \overrightarrow{\omega} = 
        \frac{\vert\overrightarrow{V}_{w1}\vert}{L}
        + \frac{\vert\overrightarrow{V}_{w2}\vert}{L}
        + \frac{\vert\overrightarrow{V}_{w3}\vert}{L}
    \end{split}
\end{equation}


\begin{gather*}
        V_{l} \angle \theta =  
        V_{w1} \angle \left(-\frac{\pi}{2}\right) 
        + V_{w2} \angle \left(\frac{2\pi}{3}-\frac{\pi}{2}\right) 
        + V_{w3} \angle \left(\frac{4\pi}{3}-\frac{\pi}{2}\right) 
\end{gather*}

\begin{align*}
    V_{l} \cos{ \theta } + jV_{l} \sin{\theta} =  
    V_{w1} \cos{ \left(-\frac{\pi}{2}\right)} + jV_{w1} \sin{ \left(-\frac{\pi}{2}\right) } \\
    + V_{w2}  \cos{ \left(\frac{\pi}{6}\right) } + jV_{w2}  \sin{ \left(\frac{\pi}{6}\right) }  \\
    + V_{w3} \cos{ \left(\frac{5\pi}{6}\right) } + jV_{w2}  \sin{ \left(\frac{5\pi}{6}\right) } 
\end{align*}

\begin{equation*}
    \begin{split}
        \omega = 
        \frac{V_{w1}}{L}
        + \frac{V_{w2}}{L}
        + \frac{V_{w3}}{L}
    \end{split}
\end{equation*}


\begin{gather}
	\begin{bmatrix} V\cdot \cos{\theta} \\  V\cdot \sin{\theta} \\  \omega \end{bmatrix}
	=
	\begin{bmatrix}
		\cos{\left(-\frac{\pi}{2}\right)} & \cos{\left(\frac{\pi}{6}\right)} & \cos{\left(\frac{5\pi}{6}\right)} \\
		\sin{\left(-\frac{\pi}{2}\right)} & \sin{\left(\frac{\pi}{6}\right)} & \sin{\left(\frac{5\pi}{6}\right)} \\
		\frac{1}{L} & \frac{1}{L} & \frac{1}{L}
	\end{bmatrix}
	\cdot
	\begin{bmatrix} V_{w1} \\  V_{w2} \\  V_{w3} \end{bmatrix}
\end{gather}


Matriz da cinemática direta.

\begin{gather}
	\begin{bmatrix}
		\cos{\left(-\frac{\pi}{2}\right)} & \cos{\left(\frac{\pi}{6}\right)} & \cos{\left(\frac{5\pi}{6}\right)} \\
		\sin{\left(-\frac{\pi}{2}\right)} & \sin{\left(\frac{\pi}{6}\right)} & \sin{\left(\frac{5\pi}{6}\right)} \\
		\frac{1}{L} & \frac{1}{L} & \frac{1}{L}
	\end{bmatrix}
	=
	\begin{bmatrix}
		0 & \sqrt{3}/2 & -\sqrt{3}/2 \\
		1 & -1/2 & -1/2  \\
		1/L & 1/L & 1/L
	\end{bmatrix}
\end{gather}



Matriz inversa.


\begin{gather}
	\begin{bmatrix} V_{w1} \\  V_{w2} \\  V_{w3} \end{bmatrix}
	=
	\begin{bmatrix}
		0 & 2/3 & L/3 \\
		-1/\sqrt{3} & -1/3 & L/3\\
		1/\sqrt{3} & -1/3 & L/3
	\end{bmatrix}
	\cdot
	\begin{bmatrix} V\cdot \cos{\theta} \\  V\cdot \sin{\theta} \\  \omega \end{bmatrix}
\end{gather}



\section{Omni Wheel}

$V_{w}$ é velocidade linear da roda, $r$ raio da roda, $\omega_{w} $ é a velocidade angular da roda.


\[V_{w1} = \omega_{w1}\cdot r \] 

