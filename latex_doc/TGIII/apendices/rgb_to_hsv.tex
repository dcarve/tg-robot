

\chapter{Transcrição da conversão de RGB para HSV em um bloco de código C \label{anx_rgb_to_hsv}}

\lstset{language=C}
\begin{lstlisting}
#include <stdio.h>
float maxOfThree(float a, float b, float c) {
	if ((a >= b) && (a >= c)) return a;
	else if ((b >= a) && (b >= c)) return b;
	else return c;
}
float minOfThree(float a, float b, float c) {
	if ((a <= b) && (a <= c)) return a;
	else if ((b <= a) && (b <= c)) return b;
	else return c;
}
void rgbToDirAngAndMag(int r, int g, int b, float *h, float *s) {
	float r_norm = r / 255.0, g_norm = g / 255.0, b_norm = b / 255.0;
	float cmax = maxOfThree(r_norm, g_norm, b_norm);
	float cmin = minOfThree(r_norm, g_norm, b_norm);
	float diff = cmax - cmin;
	if (cmax == cmin) {
		*h = 0;
	} else if (cmax == r_norm) {
		*h = fmod((60 * ((g_norm - b_norm) / diff) + 360), 360);
	} else if (cmax == g_norm) {
		*h = fmod((60 * ((b_norm - r_norm) / diff) + 120), 360);
	} else if (cmax == b_norm) {
		*h = fmod((60 * ((r_norm - g_norm) / diff) + 240), 360);
	}
	if (cmax == 0) {*s = 0;} else {*s = (diff / cmax) * 100;}
	*h = 360 - *h;
}
void main() {
	int r = 255, g = 0, b = 255; float h, s;
	rgbToDirAngAndMag(r, g, b, &h, &s);
	printf("Direction (Hue): %.2f\n", h); // res = 60
	printf("Magnitude (Saturation): %.2f\n", s); // res = 100
}
	
\end{lstlisting}