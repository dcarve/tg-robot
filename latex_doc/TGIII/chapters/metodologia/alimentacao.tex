
\section{Alimentação}


\subsection{Microcontrolador}

Para alimentação do ESP32, pode usar tanto uma bateria de 5v ou 3.3v, pois possui dois pinos para alimentação: 

\subsubsubsection{Pino de 5v não regulado}

Fontes divergem em relação à tensão máxima (\cite{esp32_reference_power_supply_1}, \cite{esp32_reference_power_supply_2}),
que pode ser usada, porém, a maioria concorda em manter de 6v a 7v.

\subsubsubsection{Pino de 3.3v regulado}
Esse pino pode receber no máximo 3.3v, podendo funcionar entre 3.1v e 3.3v sem problema.

\subsubsubsection{micro USB}
Essa opção permite usar um powerbank, porém o conector do ESP32 é um Micro-USB B, modelo que usa protocolo USB 2.0, que pode fornece apenas a 500mA a 5v \cite{micro_usb_b}.
500mA É suficiente para alimentar o ESP32, que pode consumir até 260mA \cite{esp_max_current}.
Porem um powerbank pode ser uma solução muito ineficiente do ponto de vista energético, os modelos populares possuem baterias de lítio de 3.7v, que é convertido para 5v,
e posteriormente dentro do ESP32 a tensão é convertida novamente para 3.3v. Apenas no modelo PN-952 da CNHPineng sai de 5000mAh a 3.7v  para 3160mAh a 5v,
podendo novamente perder mais corrente por hora ao ser convertido para 3.3v no ESP32


\subsection{Motores DC}
\lipsum[1]

\subsection{Motores de passo}

Os motores de passo NEMA 17 podem ser alimentados, através do driver, com 12v a 24v.
Observando a possibilidade de uma bateria de 12v/24v ficar sem uso depois do projeto,
e considerando que ferramentas elétricas possuem baterias padronizadas de 12v/18v/36V, foi considerando utilizar 3 baterias de 12v da bosch, que já estavam disponíveis para uso.
Cada bateria tem tensão nominal de 12v a 2Ah, do modelo GBA, que são usadas em aspirador de pó, esmerilhadeiras, furadeiras, plainas e serras circulares.

