

Apesar dos diversos desenvolvimentos recentes, pesquisas no campo da robótica móvel são um fenômeno ocorrendo há mais de 50 anos - segundo os padrões atuais, 
o primeiro robô móvel foi o Shakey, desenvolvido entre 1966 e 1972 \cite{TAKAHASHI}. 
Sua principal característica distintiva era a habilidade de perceber arrazoar a respeito de seu entorno, 
sendo capaz de desenvolver tarefas que requeressem planejamento, encontrar rotas e reposicionar pequenos objetos \cite{sri_international}.

À medida em que as técnicas para se construir e controlar robôs móveis (com particular interesse nos robôs móveis autônomos), 
e a isso se somando o fato de que os materiais para sua construção tornaram-se cada vez mais acessíveis (em termos de disponibilidade 
e também de redução de custos), já a partir da década de 80 começaram a surgir robôs autônomos em diversos laboratórios e centros de pesquisa;
 mais recentemente, empresas começaram a comercializar robôs para usuários domésticos, em aplicações como cortadores de grama, aspiradores e pó, 
 e mesmo robôs voltados para entretenimento \cite{TAKAHASHI}.

Robôs são classificáveis diversas maneiras, tais como forma de movimentação, os tipos de tarefas executadas e o seu grau de autonomia, 
bem como agrupando-os entre aquáticos, aéreos e terrestres. A escolha de um dado sistema de locomoção depende de diversas características do robô e 
da tarefa a ser executada, como manobrabilidade, controlabilidade, estabilidade, eficiência e tração \cite{TAKAHASHI}.

Ao se classificar robôs móveis, também é possível se empregar como critério características cinemáticas - particularmente, 
a capacidade do robô se movimentar em qualquer direção. A robôs com restrições em determinados tipos de movimento dá-se o nome de não-holonômicos, 
em oposição a robôs holonômicos, capazes de movimentação em qualquer direção (estritamente, robôs com quantidades de velocidades igual a seu grau de liberdade \cite{TAKAHASHI}).

Em se tratando de robôs terrestres, suas restrições não-holonômicas são consequência direta das rodas empregadas em sua construções. 
Rodas convencionais permitem uma quantidade de movimentos limitada, e, para contornar isso, é possível construir rodas omnidirecionais aos se acrescentar rotores à estrutura de uma roda convencional \cite{TAKAHASHI}.
